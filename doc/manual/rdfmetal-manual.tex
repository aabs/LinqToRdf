\documentclass{article}
\usepackage[dvips]{graphicx}
\author{Andrew Matthews\\
  matthews.andrew@gmail.com\\
  http://code.google.com/p/linqtordf}
\title{Using the \rm{} code generator}
\date{\today}
\newcommand{\cs}{C\texttt{\#}}
\def\spq{SPARQL}
\def\rm{\texttt{RdfMetal}}
\def\l2r{\texttt{LinqToRdf}}
\begin{document}
\maketitle

\begin{abstract}
\rm\ is part of the \l2r{} tool suite. It reads an ontology in a semantic
web database and generates \l2r{} compatible code from it. This walkthrough
shows how to use it to generate code for your project.
\end{abstract}

\section{Overview \& Usage} 

\rm{} works by querying a remote \spq{} enabled triple store to get information
about the classes defined in a user specified ontology. It stores the results in
a metadata file for use during a code generation phase. During a code generation
phase the tool will read the metadata creating \cs{} source code from it.

The two phases of process (metadata retrieval and code generation) are
independent and can be invoked separately or together. The common thread that links
them is the metadata file stored at the end of the metadata retrieval phase.

\rm{} takes these options:

\begin{verbatim}
$ ./RdfMetal.exe /?
RdfMetal  0.1.0.0 - Copyright (c) Andrew Matthews 2008
A code generator for LinqToRdf

Usage: RdfMetal [options] is used with the following options
Options:
  -e -endpoint:PARAM      The SPARQL endpoint to query.
  -h -handle:PARAM        The ontology name to be used in 
                          LinqToRdf for prefixing URIs and 
			  disambiguating class names and 
			  properties.
  -? -help                Show this help list
     -help2               Show an additional help list
  -i -ignorebnodes        Ignore BNodes. Use this if you only 
                          want to generate 
                          code for named classes.
  -m -metadata:PARAM      Where to place/get the collected 
                          metadata.
  -n -namespace:PARAM     The XML Namespace to extract classes 
                          from.
  -N -netnamespace:PARAM  The .NET namespace to place the 
                          generated source in.
  -o -output:PARAM        Where to place the generated code.
  -r -references:PARAM    A comma separated list of namespaces to 
                          reference in the generated source.
     -usage               Show usage syntax and exit
  -V -version             Display version and licensing informa-
                          tion
\end{verbatim}

If you provide a \spq{} endpoint, \rm{} will connect to it and retrieve
whatever domain model it can find there. If you provide a filename for the
metadata to be stored in, it will save the metadata as XML in that file. If you
want you can stop at that point and generate code later. If you don't provide a
metadata storage location the data will be passed directly to the code
generation part of the application, that will generate source from it. To generate
source, you must provide an output location for the source. If no output
location is provided, then no source is created and the application stops.

The steps you need to follow are:

\begin{list}{-}
\item Locate the \spq{} endpoint where the data is stored
\item Restrict the ontology namespace URI to the object model you are interested in.
\item Invoke \rm{} to retrieve the class definitions from the remote \spq{} endpoint.
\item Add the source \rm{} generates to your project
\item Reference the \texttt{LinqToRdf.dll} assembly in your project
\item Query the object model using LINQ.
\end{list}

\section*{Walkthrough}

\subsection{Installing \rm{}}
\rm{} is part of the \l2r{} tool suite. If you have the latest version of
\l2r{}, then you should have a copy. If not, you should download and install
the latest featured release of \l2r{} at
\texttt{http://code.google.com/p/} \texttt{linqtordf}. After installing \l2r{} you
should have a program files directory at \texttt{c:/Program Files/Andrew
Matthews} \texttt{/LinqToRdf}.  Inside which is all that you need to be able to run \rm{}.
It's a command line tool, so you will need to either put that directory in the
path, or refer to the \rm{} executable using the full path, as we'll do in this
walkthrough.

\subsection{Finding the endpoint URI}
The first thing you will require is the \spq{} endpoint URL of the semantic web
database you're planning to work with. For this walkthrough, we'll be using the
sample triple store that is used by the \l2r{} project for testing. In this
case, you query it through the URL
\texttt{http://localhost/linqtordf} \texttt{/SparqlQuery.aspx}. You can find out more
about hosting your own ontologies in the \l2r{} manual, or in my article
Understanding \spq{}.

You need to know this endpoint URL because \rm{} will send a series of
\spq{} queries to the database requesting information about the classes stored
there and their properties and relationships.

\subsection{Create a pre-build step in your project}

In this walkthrough, we use \rm{} to generate code as part of a pre-build
step. That means the metadata will be retrieved and used to generate source
code every time your project gets built. This probably overkill, especially if
the remote ontology is not subject to frequent changes. In that case you might
want to perform these steps manually as required.

This command line invokes \rm{} using the build step macro \$(ProjectDir)
defined in Visual Studio.  Typically, you define the \spq{} endpoint as an
initial phase, and then cache metadata gathered in an XML file. Once the
metadata has been gathered, \cs{} source code can be generated from it. If the
target ontology is static, you may choose not to continue using \rm{} after the
initial source code generation. Here, we're simply gathering the metadata every
time and storing it in a source file called \texttt{music.cs}.

\begin{verbatim}
"c:\Program Files\Andrew Matthews\RdfMetal\RdfMetal.exe"
    -e:http://localhost/linqtordf/SparqlQuery.aspx
    -i -n http://aabs.purl.org/ontologies/2007/04/music\#
    -o "$(ProjectDir)music.cs" -N:RdfMetal.Music -h music
\end{verbatim}

\section{Using \rm{} with Visual Studio}
\rm{}  was designed to used from a pre-build event in Visual Studio. The example
above, shows how it can be used to generate source for the small ontology used
to test \l2r{}.  It queries a local website called \texttt{LinqToRdf}. (You
can get instructions on how to set up such a self hosted ontology from the
\l2r{} manual.) I'm restricting the classes to only those defined in the
\texttt{http://aabs.purl.org/ontologies/2007/04/music\#} namespace and am
generating code in the \texttt{RdfMetal.Music} .NET namespace. The internal name
I'm using is '\texttt{music}', and that's the preferred prefix to be used in any generated RDF or \spq{}.

When \rm{} is run it outputs a list of the classes that it is generating source
for, then writes 'done'. Here's some output from the music ontology:

\begin{verbatim}
------ Build started: Project: RdfMetalTestHarness
C:\. . .\linqtordf\src\RdfMetal\bin\Debug\RdfMetal.exe
  -e:http://localhost/LINQTORDF/SparqlQuery.aspx -i -n
  http://aabs.purl.org/ontologies/2007/04/music\# -o
  "C:\. . .\linqtordf\src\unit-testing\
  RdfMetalTestHarness\music.xml" -N:RdfMetal.Music -h music

ProducerOfMusic
SellerOfMusic
NamedThing
TemporalThing
Person
Band
Studio
Music
Album
Track
Song
Mp3File
Genre
done.
\end{verbatim}

Each of these classes is defined in the \texttt{store.n3} file in the unit tests
directory in the \l2r{} solution. The source that it generates will be in the
file \texttt{music.cs}. The output is too long and repetitive to be worth including in
full, but here's some edited highlights. The generator creates a \texttt{DataContext}
class containing standard query properties for each of the class types found in
the metadata extraction process. In this case the queries are included for
\texttt{Album} and \texttt{Track} 

\begin{verbatim}
  public class musicDataContext : RdfDataContext
  {
    public musicDataContext(TripleStore store)
      : base(store)
    {
    }

    public musicDataContext(string store)
      : base(new TripleStore(store))
    {
    }

    public IQueryable<Album> Albums
    {
      get { return ForType<Album>(); }
    }

    public IQueryable<Track> Tracks
    {
      get { return ForType<Track>(); }
    }

\end{verbatim}

In most cases the classes generated are empty, but in the test data the Track
and Album classes have several DatatypeProperties as well as ObjectProperties.
Here's the code generated for the \texttt{Track} class.

\begin{verbatim}
[OwlResource(OntologyName = "music", 
             RelativeUriReference = "Track")]
public class Track : OwlInstanceSupertype
{
    #region Datatype properties

    [OwlResource(OntologyName = "music",
                 RelativeUriReference = "title")]
    public string title { get; set; } // Track
    [OwlResource(OntologyName = "music", 
                 RelativeUriReference = "artistName")]
    public string artistName { get; set; } // Track
    [OwlResource(OntologyName = "music", 
                 RelativeUriReference = "albumName")]
    public string albumName { get; set; } // Track
    [OwlResource(OntologyName = "music", 
                 RelativeUriReference = "genreName")]
    public string genreName { get; set; } // Track
    [OwlResource(OntologyName = "music", 
                 RelativeUriReference = "comment")]
    public string comment { get; set; } // Track
    [OwlResource(OntologyName = "music", 
                 RelativeUriReference = "fileLocation")]
    public string fileLocation { get; set; } // Track

    #endregion

    #region Incoming relationships properties

    #endregion

    #region Object properties

    [OwlResource(OntologyName = "music", 
                 RelativeUriReference = "isTrackOn")]
    public string isTrackOnUri { get; set; }

    private EntityRef<Album> _isTrackOn { get; set; }

    [OwlResource(OntologyName = "music", 
                 RelativeUriReference = "isTrackOn")]
    public Album isTrackOn
    {
        get
        {
            if (_isTrackOn.HasLoadedOrAssignedValue)
                return _isTrackOn.Entity;
            if (DataContext != null)
            {
               var ctx = (musicDataContext) DataContext;
               _isTrackOn = new EntityRef<Album>(
                     from x in ctx.Albums 
                     where x.HasInstanceUri(isTrackOnUri) 
                     select x);
               return _isTrackOn.Entity;
            }
            return null;
        }
    }

    #endregion
}

\end{verbatim}

Note the use of the \texttt{EntityRef} in the \texttt{\_isTrackOn} field and the
\texttt{isTrackOn} property to provide navigation across the object graph.

\begin{verbatim}
[OwlResource(OntologyName = "Music", 
             RelativeUriReference = "Album")]
public class Album : OwlInstanceSupertype
{
  public Album()
  {
  }

  [OwlResource(OntologyName = "Music", 
               RelativeUriReference="name")]
  public string Name { get; set; }

  private EntitySet<Track> _Tracks = new EntitySet<Track>();
  [OwlResource(OntologyName = "Music", 
               RelativeUriReference = "isTrackOn")]
  public EntitySet<Track> Tracks
  {
    get
    {
      if (_Tracks.HasLoadedOrAssignedValues)
        return _Tracks;
      if (DataContext != null)
      {
        _Tracks.SetSource(from t in 
          ((MusicDataContext)DataContext).Tracks 
           where t.AlbumName == Name 
           select t);
      }
      return _Tracks;
    }
  }
}

\end{verbatim}

Again notice the use of \texttt{EntitySet}s to provide navigation. This time the
navigation is from parent to child.

Once the code is generated all you need to do is incorporate it into your
project and use it like any other \l2r{} code. For instructions on how to do
that, consult the \l2r{} manual for guidance.

\end{document}
